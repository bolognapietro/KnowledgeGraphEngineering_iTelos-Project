\section{Introduction}

\noindent Access to sports facilities and related events is essential for enhancing the quality of life in modern cities and regions. In areas like Trentino Province, providing convenient access to a range of sports infrastructure and events is increasingly important for both residents and visitors. Promoting an active lifestyle and supporting community-driven initiatives are key to strengthening the livability of the region.\\

\noindent To address these needs, we present a project for the Knowledge Graph Engineering course, integrating data on sports facilities and events in Trentino. This Knowledge Graph will offer citizens, tourists, and local authorities a comprehensive, interconnected view of available sports facilities, and related sport events. By combining data on facilities and events, the Knowledge Graph will empower users to make informed decisions that enhance community participation, public health, and the region’s sporting culture.\\

\noindent Reusability is one of the main principles in the Knowledge Graph Engineering (KGE) process defined by iTelos. The KGE project documentation plays an important role to enhance the reusability of the resources handled and produced during the process. A clear description
of the resources as well as of the process (and single activities) developed, provides a clear understanding of the project, thus serving such an information to external readers for the future exploitation of the project’s outcomes.\\

\noindent The current document aims to provide a detailed report of the project developed following the iTelos methodology. The report is structured as follows:
\begin{itemize}
    \item Section 2: Definition of the project's purpose and its domain of interest.

    \item Section 3: High level description of the project development, based on the Produce role's objectives.  
    
    \item Sections 4, 5, 6, 7 and 8: The description of the iTelos process phases and their activities, divided by knowledge and data layer activities.
    
    \item Section 9: The description of the evaluation criteria and metrics applied to the project final outcome.

    \item Section 10: The description of the metadata produced for all (and all kind of) the resources handled and generated by the iTelos process, while executing the project.

    \item Section 11: Conclusions and open issues summary.
\end{itemize}

\noindent You can access the GitHub repository, which contains all the materials used during the project's development, via this \href{https://github.com/christiansassi/knowledge-graph-engineering-project}{link}.